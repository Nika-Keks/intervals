\documentclass[a4paper,12pt]{article}

\usepackage[hidelinks]{hyperref}
\usepackage{amsmath}
\usepackage{mathtools}
\usepackage{shorttoc}
\usepackage{cmap}
\usepackage[T2A]{fontenc}
\usepackage[utf8]{inputenc}
\usepackage[english, russian]{babel}
\usepackage{xcolor}
\usepackage{graphicx}
\usepackage{float}
\graphicspath{{./../}}

\definecolor{linkcolor}{HTML}{000000}
\definecolor{urlcolor}{HTML}{0085FF}
\hypersetup{pdfstartview=FitH,  linkcolor=linkcolor,urlcolor=urlcolor, colorlinks=true}

\DeclarePairedDelimiter{\floor}{\lfloor}{\rfloor}

\renewcommand*\contentsname{Содержание}

\newcommand{\plot}[3]{
    \begin{figure}[H]
        \begin{center}
            \includegraphics[scale=0.6]{./doc/img/#1.png}
            \caption{#2}
            \label{#3}
        \end{center}
    \end{figure}
}

\begin{document}
    \begin{titlepage}
	\begin{center}
		{\large Санкт-Петербургский политехнический университет\\Петра Великого\\}
	\end{center}
	
	\begin{center}
		{\large Физико-механический иститут}
	\end{center}
	
	
	\begin{center}
		{\large Кафедра «Прикладная математика»}
	\end{center}
	
	\vspace{8em}
	
	\begin{center}
		{\bfseries Отчёт по лабораторной работе №2 \\по дисциплине «Анализ данных с интервальной неопределённостью»}
	\end{center}
	
	\vspace{5em}
	
	\begin{flushleft}
		\hspace{16em}Выполнил студент:\\\hspace{16em}Аникин Александр Алексеевич\\\hspace{16em}группа: 5040102/20201
		
		\vspace{2em}
		
		\hspace{16em}Проверил:\\\hspace{16em}к.ф.-м.н., доцент\\\hspace{16em}Баженов Александр Николаевич
		
	\end{flushleft}
	
	
	\vspace{6em}
	
	
	\begin{center}
		Санкт-Петербург\\2023 г.
	\end{center}	
	
\end{titlepage}
    \newpage

    \tableofcontents
    \listoffigures
    \newpage

    \section{Постановка задачи}
    \section{Теория}
    \subsection{Точечная линейная регрессия}
    \quad Рассматривается задача восстановления зависимости для выборки
    $ (X, \textbf(Y))$, $ X = \{x_i\}_{i=1}^{n}, \textbf{Y} = \{\textbf{y}_i\}_{i=1}^{n} $,
    $ x_i $ - точеный, $ \textbf{y}_i $ - интервальный.
    Пусть искомая модель задана в классе линейных функций

    \label{e:model}
    \begin{equation}
        y = \beta_0 + \beta_1 x
    \end{equation}

    Поставим задачу оптимизацию \ref{e:task} для нахождения точечных оценок
    параметров $ \beta_0, \beta_1 $.

    \label{e:task}
    \begin{equation}
        \begin{gathered}
            \sum_{i = 1}^{m}w_{i} \to \min \\
            \text{mid}\textbf{y}_{i} - w_{i} \cdot \text{rad}\textbf{y}_{i} \leq X\beta \leq \text{mid}\textbf{y}_{i} + w_{i} \cdot \text{rad}\textbf{y}_{i} \\
            w_{i} \geq 0, i = 1, ..., m \\
            w, \beta - ?
        \end{gathered}
    \end{equation}
    
    Задачу \ref{e:task} можно решить методами линейного программирования.

    \subsection{Информационное множество}
    \quad \textsl{Информационным множеством} задачи восстановления зависимости
    будем называть множество значений всех параметров зависимости,
    совместных с данными в каком-то смысле. 

    \textsl{Коридором совместных зависимостей} задачи восстановления зависимости
    называется многозначное множество отображений $ \Upsilon $, сопоставляющее
    каждому значению аргумента $ x $ множество
    
    \begin{equation}
        \Upsilon(x) = \bigcup_{\beta \in \Omega} f(x, \beta)
    \end{equation}

    , где $ \Omega $ - информационное множество, $ x $ - вектор переменных, $ \beta $ - вектор оцениваемых параметров. 

    Информационное множество может быть построено, как пересечение полос, заданных
    
    \begin{equation}
        \underline{\textbf{y}_i} \leq \beta_0 + \beta_1 x_{i1} + ... + \beta_m x_{im} \leq \overline{\textbf{y}_i}
    \end{equation}
    , где $ i = \overline{1, n} \textbf{y}_i \in \textbf{Y}, x_i \in X $, $ X $ - точечная выборка переменных,
    $ \textbf{Y} $ - интервальная выборка откликов.

    \section{Результаты}
    \quad Данные $ S_X $ были взяты из файлов \textsl{data/dataset1/X/X\_0.txt}, \newline
    где $ X \in \{-0\_5, -0\_25, +0\_25, +0\_5 \} $.
    Набор $ \delta_i $ получен из соответствующих файлов в \textsl{data/dataset1/ZeroLine.txt}.

    Набор значений $ X $ точечный и одинаков для всех выборок. \newline
    $ X = [-0.5, -0.25, 0.25, 0.5]  $.
    Набор значений отклика $ Y $ интервальный и разный для каждой выборки.
    
    Построим линейную регрессию и найдём информационное множество для нескольких выборок.

    Рассмотрим первую выборку $ Y_1 $.
    $ Y_1 $ следующим образом. $ y_i = [\min_{t \in S_i}{S_i}, \max_{t \in S_i}{S_i}]$,
    $ i = [-0.5, -0.25, +0.25, +0.25], y_i \in Y_1 $.
    \plot{Y1}{Первая выборка, $ Y_1 $}{p:sampleY1}

    % Индекс Жаккара первой выборки равен $ JK(Y_1) = 0.03001 $ (в этой работе $ JK(X) \in [0, 1] $).

    Построим линейную регрессию, решив задачу \ref{e:task} для выборки $ Y_1 $.
    \plot{PointRegressionY1}{Точечная линейная регрессия для $ Y_1 $}{p:regressionY1}
    
    Получим следующие оценки для параметров: $ \beta_0 = 0.00076, \beta_1 = 0.86426 $.
    Тогда полученная модель имеет вид $ y = 0.00076 + 0.86426 x $.

    Найдём для данной выборки информационное множество.
    \plot{InformSetY1}{Информационное множество для $ Y_1 $}{p:informSetY1}

    На рис. \ref{p:informSetY1} можно заметит, что найденные параметры
    $ \beta_0, \beta_1 $ решением задачи \ref{e:task} лежат внутри информационного множества.

    Построим коридор совместных значений для выборки $ Y_1 $ и информационного множества \ref{p:informSetY1}
    и оценим значения выходной переменной $ y $ вне пределов значений входной переменной $ x $.
    \plot{InformSetCorridorY1}{Коридор совместных значений для $ Y_1 $}{p:informSetCorridorY1}

    На рис. \ref{p:informSetCorridorY1} видно, что построенная точечная регрессия лежит внутри коридора совместных значений,
    что согласуется с рис. \ref{p:informSetY1}.

    Проведём аналогичные построения для выборки $ Y_2 $, построенную следующим образом.
    $ y_i = [median(S_i) - \varepsilon, median(S_i) + \varepsilon] $, $ \varepsilon = \frac{1}{2^{14}} $
    $ i = [-0.5, -0.25, +0.25, +0.25], y_i \in Y_2 $.
    $ Y_2 $ имеет вид.
    \plot{Y2}{Вторая выборка, $ Y_2 $}{p:sampleY2}
    
    % Индекс Жаккра $ Y_2 $ равен $ JK(Y_2) = 0.0141 $.

    Построим точечную линейную регрессию для $ Y_2 $.
    \plot{PointRegressionY2}{Точечная линейная регрессия для $ Y_2 $}{p:regressionY2}

    Для $ Y_2 $ получили следующие оценки параметров: $ \beta_0 = 0.0005, \beta_1 = 0.85324 $.
    Построим информационное множество и коридор совместных значений для $ Y_2 $.

    \plot{InformSetY2}{Информационное множество для $ Y_2 $}{p:informSetY2}
    \plot{InformSetCorridorY2}{Коридор совместных значений для $ Y_2 $}{p:informSetCorridorY2}

    В итоге для $ Y_2 $ получили, что точечная регрессия также попала в информационное множество.

    Теперь проведём аналогичные построения для $ Y_3 $, построенную аналогично $ Y_1 $,
    за исключением отсутствия учёта $ \delta_i $.
    $ Y_3 $ имеет вид.
    \plot{Y3}{Третья выборка, $ Y_3 $}{p:sampleY3}

    Построим точечную регрессию.
    \plot{PointRegressionY3}{Точечная линейная регрессия для $ Y_3 $}{p:regressionY3}

    Для $ Y_3 $ точечная линейная регрессия дала следующие оценки: $ \beta_0 = -0.0052, \beta_1 = 0.85169 $.
    Информационное множество и коридор совместных значений имеют следующий вид.

    \plot{InformSetY3}{Информационное множество для $ Y_3 $}{p:informSetY3}
    \plot{InformSetCorridorY3}{Коридор совместных значений для $ Y_3 $}{p:informSetCorridorY3}

    \section{Обсуждение}
    \quad Из полученных результатов можно заметить следующее.
    Наиболее маленькое информационное множество было получено для выборки $ Y_2 $,
    что неудивительно, так как $ Y_2 $ имеет наименьшую интервальную неопределённость.
    Соответственно для $ Y_2 $ получили и наиболее узкий коридор совместных значений.


    \begin{table}[H]
            \begin{tabular}{| c | c | c |}
                \hline
                0 & $ \beta_0 $ & $ \beta_1 $ \\
                \hline
                $ Y_1 $ & $ 0.00076 $ & $ 0.86426 $ \\
                \hline
                $ Y_2 $ & $ 0.0005 $ & $ 0.85324 $ \\
                \hline
                $ Y_3 $ & $ -0.0052 $ & $ 0.85169 $ \\
                \hline
            \end{tabular}
            \centering
    \end{table}
    

    Видно, что для выборок $ Y_1, Y_2 $ точечная линейная регрессия дала более точный результат,
    близкий к ожидаемому $ \beta_0 = 0.0, \beta_1 = 1.0 $. Для $ Y_3 $ получили более неточную оценку,
    так оценка параметра $ \beta_0 $ для $ Y_3 $ отличается на порядок от соответствующей оценки для $ Y_1, Y_2 $.

    Также стоит отметить, что во всех случаях точечная линейная регрессия попала в информационное множество.
\end{document}
