\documentclass[a4paper,12pt]{article}

\usepackage[hidelinks]{hyperref}
\usepackage{amsmath}
\usepackage{mathtools}
\usepackage{shorttoc}
\usepackage{cmap}
\usepackage[T2A]{fontenc}
\usepackage[utf8]{inputenc}
\usepackage[english, russian]{babel}
\usepackage{xcolor}
\usepackage{graphicx}
\usepackage{float}
\graphicspath{{./../}}

\definecolor{linkcolor}{HTML}{000000}
\definecolor{urlcolor}{HTML}{0085FF}
\hypersetup{pdfstartview=FitH,  linkcolor=linkcolor,urlcolor=urlcolor, colorlinks=true}

\DeclarePairedDelimiter{\floor}{\lfloor}{\rfloor}

\renewcommand*\contentsname{Содержание}

\newcommand{\plot}[3]{
    \begin{figure}[H]
        \begin{center}
            \includegraphics[scale=0.6]{./doc/img/#1.png}
            \caption{#2}
            \label{#3}
        \end{center}
    \end{figure}
}

\begin{document}
    \begin{titlepage}
	\begin{center}
		{\large Санкт-Петербургский политехнический университет\\Петра Великого\\}
	\end{center}
	
	\begin{center}
		{\large Физико-механический иститут}
	\end{center}
	
	
	\begin{center}
		{\large Кафедра «Прикладная математика»}
	\end{center}
	
	\vspace{8em}
	
	\begin{center}
		{\bfseries Отчёт по лабораторной работе №2 \\по дисциплине «Анализ данных с интервальной неопределённостью»}
	\end{center}
	
	\vspace{5em}
	
	\begin{flushleft}
		\hspace{16em}Выполнил студент:\\\hspace{16em}Аникин Александр Алексеевич\\\hspace{16em}группа: 5040102/20201
		
		\vspace{2em}
		
		\hspace{16em}Проверил:\\\hspace{16em}к.ф.-м.н., доцент\\\hspace{16em}Баженов Александр Николаевич
		
	\end{flushleft}
	
	
	\vspace{6em}
	
	
	\begin{center}
		Санкт-Петербург\\2023 г.
	\end{center}	
	
\end{titlepage}
    \newpage

    \tableofcontents
    \listoffigures
    \newpage

    \section{Постановка задачи}

    \quad Решить задачу восстановления зависимости на экспериментальных данных.
    Данные взять из материалов (таблица) статьи Roederer, Ian U., et al. "Element abundance patterns in stars indicate fission of nuclei heavier than uranium." Science 382.6675 (2023).

    \section{Теория}
    \subsection{Точечная линейная регрессия}
    \quad Рассматривается задача восстановления зависимости для выборки
    $ (X, \textbf(Y))$, $ X = \{x_i\}_{i=1}^{n}, \textbf{Y} = \{\textbf{y}_i\}_{i=1}^{n} $,
    $ x_i $ - точечный, $ \textbf{y}_i $ - интервальный.
    Пусть искомая модель задана в классе линейных функций

    \label{e:model}
    \begin{equation}
        y = \beta_0 + \beta_1 x
    \end{equation}

    Поставим задачу оптимизации \ref{e:task} для нахождения точечных оценок
    параметров $ \beta_0, \beta_1 $.

    \label{e:task}
    \begin{equation}
        \begin{gathered}
            \sum_{i = 1}^{m}w_{i} \to \min \\
            \text{mid}\textbf{y}_{i} - w_{i} \cdot \text{rad}\textbf{y}_{i} \leq X\beta \leq \text{mid}\textbf{y}_{i} + w_{i} \cdot \text{rad}\textbf{y}_{i} \\
            w_{i} \geq 0, i = 1, ..., m \\
            w, \beta - ?
        \end{gathered}
    \end{equation}
    
    Задачу \ref{e:task} можно решить методами линейного программирования.

    \subsection{Информационное множество}
    \quad \textsl{Информационным множеством} задачи восстановления зависимости
    будем называть множество значений всех параметров зависимости,
    совместных с данными в каком-то смысле. 

    \textsl{Коридором совместных зависимостей} задачи восстановления зависимости
    называется многозначное множество отображений $ \Upsilon $, сопоставляющее
    каждому значению аргумента $ x $ множество
    
    \begin{equation}
        \Upsilon(x) = \bigcup_{\beta \in \Omega} f(x, \beta)
    \end{equation}

    , где $ \Omega $ - информационное множество, $ x $ - вектор переменных, $ \beta $ - вектор оцениваемых параметров. 

    Информационное множество может быть построено как пересечение полос, заданных
    
    \begin{equation}
        \underline{\textbf{y}_i} \leq \beta_0 + \beta_1 x_{i1} + ... + \beta_m x_{im} \leq \overline{\textbf{y}_i}
    \end{equation}
    , где $ i = \overline{1, n} \textbf{y}_i \in \textbf{Y}, x_i \in X $, $ X $ - точечная выборка переменных,
    $ \textbf{Y} $ - интервальная выборка откликов.

    \section{Результаты}
    \quad Данные были взяты из таблицы S1 приложенной к работе Roederer, Ian U., et al. "Element abundance patterns in stars indicate fission of nuclei heavier than uranium." Science 382.6675 (2023)

    В ней представлены определенные значения для различных элементов и их стандартные отклонения.
    В качестве $x$ выступает значение $[Eu/Fe]$, а в качестве $y$ $X \pm eX,\ X \in \{Se, Sr, Y, Nb, Mo\}$.
    Эти данные соответсвуют части данных с \textit{рис Fig. 2. (A) Abundance ratios of groups of elements that do or do not correlate with [Eu/Fe]. }
    Остальные же даные в материалах статьи представлены рядом ссылок, и автоматизировать их сбор не представляется возможным.

    Далее рассмотрим сами данные и получившие результаты.

    \plot{Se Sr Y Nb Mo}{
        Синим цветом представлены значения для различных элементов с их стандартными отклонениями.
        Оранжевым цветом показаны результаты регрессии.
    }{linreg}

    
    Полученны $(\beta_1, \beta_2) = (-0.0, -0.9)$. Полученные результаты отличаются от результатов в статье из-за 
    отстутствия сдвига, однако общий вывод будет одинаков из-за $\beta_1$ близкого к нулю.

    Также стоит отметить что информационное множество пустое, так как настоящая зависимость отличается от линейной
    и сами интвервалы представляют собой лишь стандартное отклонение от среднего.

    Чтобы сделать выборку совместной, произведем обинтервваливание следующим образом:
    вместо $(a, b) = (\mu - \sigma, \mu + \sigma)$ по правилу трех сигм будем брать 
    $(a, b) = (\mu - 3\sigma, \mu + 3\sigma)$, что раширит интервалы и при нормальном распределении
    значения окажутся внутри интервала с вероятностью $0.9973$ вместо $0.6826$
    для радиуса в одно стандартное отклонение. Самое главное, что такое обитеграливание 
    позволин сделать выбокру совместной, а информационное множество не пустым.

    Таким образом получаем следующую выборку:

    \plot{Se Sr Y Nb Mo 3sig}
    {Синим цветом представлены значения для различных элементов с тремя стандартными отклонениями.
    Оранжевым цветом показаны результаты регрессии.}
    {linreg-3sig}

    В этом случае результаты регрессии не поменяются $(\beta_1, \beta_2) = (-0.0, -0.9)$, 
    а информационное множество будет выглядеть следующим образом:

    \plot{Se Sr Y Nb Mo 3sig infoset}
    {Информационное множество для выборки с радиусом в $3\sigma$}
    {infoset-3sig}

    Итого получим следующие интервалы для $(\beta_1, \beta_2) = ([0.093,\ 0.392], [-0.983,\ -0.755])$.
    А центр информационного множества будет следующим $(\beta_1, \beta_2) = (0.243, -0.869)$. 

    \section{Обсуждение}
    \quad Можно сказать, что полученные результаты соответствуют результатам из исходной работы 
    с учетом отсутствия сдвига и меньшего количества данных. Общий вывод об отсутствии линейной 
    зависимости между значениями сохраняется.

\end{document}
